\documentclass[12pt, letter]{article}
\usepackage[ebgaramond]{fontsetup}
    % \usepackage[lining,semibold,scaled=1.05]{ebgaramond} % font
    \usepackage[T1]{fontenc}
    \usepackage{fullpage}
    \usepackage{amsmath}
    \usepackage{amssymb}
    \usepackage{textcomp}
    \usepackage[utf8]{inputenc}
    \usepackage[margin=1in]{geometry}
    \usepackage{fontawesome}
    \usepackage{etoolbox}
    \usepackage{booktabs}
    \usepackage{array}

    \usepackage{latexsym}
    \usepackage{titlesec}
    \usepackage{marvosym}
    \usepackage[usenames,dvipsnames]{color}
    \usepackage{verbatim}
    \usepackage{enumitem}
    \usepackage{amssymb}
    \usepackage[hidelinks]{hyperref}
    \usepackage{fancyhdr}
    \usepackage[english]{babel}
    \input{glyphtounicode}
    \textheight=11in
    \pagestyle{empty}
    \raggedright


% page format commands %

\pagestyle{fancy}
\fancyhf{} % clear all header and footer fields
\fancyfoot{}
\renewcommand{\headrulewidth}{0pt}
\renewcommand{\footrulewidth}{0pt}

% column space
\setlength{\tabcolsep}{52pt}

% Adjust margins
\addtolength{\oddsidemargin}{-0.5in}
\addtolength{\evensidemargin}{-0.5in}
\addtolength{\textwidth}{1in}
\addtolength{\topmargin}{-.5in}
\addtolength{\textheight}{1.0in}

\urlstyle{same}

\raggedbottom
\raggedright
\setlength{\tabcolsep}{0in}

\newcolumntype{R}[1]{>{\raggedleft\arraybackslash}p{#1}} % Define a new right-aligned column type
\newcolumntype{L}[1]{>{\raggedright\arraybackslash}p{#1}} % Define a new left-aligned (no justification) column type
\newcolumntype{C}[1]{>{\centering\arraybackslash}p{#1}} % Define a new centred column type

% Sections formatting
\titleformat{\section}{
  \vspace{-4pt}\scshape\raggedright\large
}{}{0em}{}[\color{black}\titlerule \vspace{-5pt}]

% Ensure that generate pdf is machine readable/ATS parsable
\pdfgentounicode=1

\def\bull{\vrule height 0.8ex width .7ex depth -.1ex }

%-------------------------
% Custom commands
\newcommand{\resumeItem}[1]{
  \item\small{
    {#1 \vspace{-2pt}}
  }
}

\newcommand{\resumeSubheading}[4]{
  \vspace{-2pt}\item
    \begin{tabular*}{0.97\textwidth}[t]{l@{\extracolsep{\fill}}r}
      \textbf{#1} & #2 \\
      \textit{\small#3} & \textit{\small #4} \\
    \end{tabular*}\vspace{-7pt}
}

\newcommand{\resumeSubSubheading}[2]{
    \item
    \begin{tabular*}{0.97\textwidth}{l@{\extracolsep{\fill}}r}
      \textit{\small#1} & \textit{\small #2} \\
    \end{tabular*}\vspace{-7pt}
}

\newcommand{\resumeProjectHeading}[2]{
    \item
    \begin{tabular*}{0.97\textwidth}{l@{\extracolsep{\fill}}r}
      \small#1 & #2 \\
    \end{tabular*}\vspace{-7pt}
}

\newcommand{\resumeSubItem}[1]{\resumeItem{#1}\vspace{-4pt}}

\renewcommand\labelitemii{$\vcenter{\hbox{\tiny$\bullet$}}$}

\newcommand{\resumeSubHeadingListStart}{\begin{itemize}[leftmargin=0.15in, label={}]}
\newcommand{\resumeSubHeadingListEnd}{\end{itemize}}
\newcommand{\resumeItemListStart}{\begin{itemize}[noitemsep]} %[topsep=0pt,itemsep=0pt,parsep=0pt,before=\vspace{0cm},after=\vspace{1mm}] % added option to save space
\newcommand{\resumeItemListEnd}{\end{itemize}\vspace{-16pt}}

%-------------------------------------------

% DEFINITIONS FOR RESUME %%%%%%%%%%%%%%%%%%%%%%%

\newcommand{\area} [2] {
    \vspace*{-9pt}
    \begin{verse}
        \textbf{#1}   #2
    \end{verse}
}

\newcommand{\lineunder} {
    \vspace*{-8pt} \\
    \hspace*{-18pt} \hrulefill \\
}

\newcommand{\header} [1] {
    {\hspace*{-18pt}\vspace*{6pt} \textsc{#1}}
    \vspace*{-6pt} \lineunder
}

\newcommand{\employer} [3] {
    { \textbf{#1} (#2)\\ \underline{\textbf{\emph{#3}}}\\  }
}

\newcommand{\contact} [3] {
    \vspace*{-10pt}
    \begin{center}
        {\Huge \scshape {#1}}\\
        #2 \\ #3
    \end{center}
    \vspace*{-8pt}
}

\newenvironment{achievements}{
    \begin{list}
        {$\bullet$}{\topsep 0pt \itemsep -2pt}}{\vspace*{4pt}
    \end{list}
}

\newcommand{\schoolwithcourses} [4] {
    \textbf{#1} #2 $\bullet$ #3\\
    #4 \\
    \vspace*{5pt}
}

\newcommand{\school} [4] {
    \textbf{#1} #2 $\bullet$ #3\\
    #4 \\
}
% END RESUME DEFINITIONS %%%%%%%%%%%%%%%%%%%%%%%

\usepackage{garamondlibre}

\begin{document}

    

%==== HEADER ====%
\vspace*{-14pt}
\begin{center}
	{\Huge \scshape {Alex Diaz}}\\
	\vspace{1mm}
	\faMapMarker \hspace{.5mm} Atlanta, GA $\cdot$ 
	\faEnvelope \hspace{.5mm} \href{mailto:alejandrojsdiaz@outlook.com}{alejandrojsdiaz@outlook.com} $\cdot$ \faMobile \hspace{.5mm} 678-315-1715
		
	\faGithub \hspace{.5mm} \href{https://github.com/calmcoconut}{GitHub} $\cdot$
	\faLinkedin \hspace{.5mm} \href{https://www.linkedin.com/in/diazjalejandro/}{LinkedIn} $\cdot$
	\faTwitter \hspace{.5mm} \href{https://twitter.com/greetingsfriend}{Twitter} $\cdot$
    \faBriefcase \hspace{.5mm} \href{https://calmcoconut.github.io/diasDiaz/}{Diasdiaz}
    \\
\end{center}

\vspace{-14pt}
%==== Education ====%
\section{Education}
  \resumeSubHeadingListStart
    \resumeSubheading
      {Georgia Institute of Technology}{Atlanta, GA}
      {Master of Science in Computer Science \textbf{GPA 4.0 / 4.0}}{Jan. 2021 -- May 2023}
    \resumeSubheading
      {The University of Georgia}{Athens, GA}
      {Bachelor of Arts in Economics, Minor in History}{Aug. 2013 -- May 2017}
 \resumeSubHeadingListEnd

%==== Independent projects ====%
\section{Projects}

\begin{itemize}
  \item {\small \textbf{\textit{2021 -- Multi-threaded socket server and client} --} Programmed a multi-threaded server and client in C utilizing the Linux system call api and Sockets library. Demonstrated IPv4 / IPv6 interoperability and multi-threading management (C libraries).}
  \item {\small \textbf{\textit{2021 -- AI Agent} --} Architected an AI agent based on Newell's SOAR to solve to solve Raven's Progressive Matrices. I used computer vision (i.e., scale-invariant feature transformation (SIFT)) and human-like cognition algorithms to enable the agent to process IQ test images as problems; resulting in a success rate of 70\%. I implemented the agent in Python, using object oriented computer programming, OpenCV, and NumPy. Written with meta-cognition facilities to better solve problems.}
  \item {\small \textbf{\textit{2020 -- TimeTime} --}  Coded a Java Android application that prompts users through their lock screen to log their daily activities. The application uses a model-view-controller pattern to provide users an infinite scrolling list and enable interactions with good performance and a beautiful UI / UX created with Jetpack. My architecture allowed for local persistent data (RoomDB) and concurrency.}
  \item {\small \textbf{\textit{2019 - Spotify Music Advisor} --} Engineered a Java application using the Spring boot framework, capable of REST API communication with Spotify. I implemented the application to use OAuth. Enables users to query playlists.}
\end{itemize}

% \textbf{Multi-threaded Socket Program} | \small\textit{Python, NumPy, PIL, Docker, OpenCV}
% \begin{itemize}[noitemsep,topsep=0pt]
%   \item Undertook implementing an AI that solves the Raven's Progressive Matrix IQ
%   \item Demonstrated performance and problem-solving creativity comparable to a human test taker
%   \item Used \textbf{Computer Vision} algorithms via \textbf{OpenCV} and \textbf{NumPy}
% \end{itemize}

% \textbf{IQ AI agent} | \small\textit{Python, NumPy, PIL, Docker, OpenCV}
% \begin{itemize}[noitemsep,topsep=0pt]
%   \item Undertook implementing an AI that solves the Raven's Progressive Matrix IQ
%   \item Demonstrated performance and problem-solving creativity comparable to a human test taker
%   \item Used \textbf{Computer Vision} algorithms via \textbf{OpenCV} and \textbf{NumPy}
% \end{itemize}

% \textbf{TimeTime} Android Application | \small\textit{Java, Jetpack, Android Material, Room DB, Java Thread}
% \\ \href{https://github.com/calmcoconut/TimeTime}{\textit{github.com/calmcoconut/TimeTime}}
% \begin{itemize}[noitemsep,topsep=0pt]
%   \item Created a time logging Android Application following a MVC design pattern supported by Android's Room persistent database design
% \end{itemize}

% \textbf{Spotify Music Advisor} | \small\textit{Java, Spring, HTTP, OAuth2, JSON}
% \\ \href{https://github.com/calmcoconut/Music_Advisor_spotify_api}{\textit{github.com/calmcoconut/Music\_Advisor\_spotify\_api}}
% \begin{itemize}[noitemsep,topsep=0pt]
%   \item Implemented a RESTful server capable of OAuth handshake and authentication with Spotify
% \end{itemize}

\vspace{-20pt}
 %-----------EXPERIENCE-----------
\section{Experience}

\resumeSubHeadingListStart
  \resumeSubheading
    {Software Engineer Tutor}{Feb 2020 -- Present}
    {Tokyo Coding Club \& FDS Services}{Tokyo, Japan}
    \resumeItemListStart
      \resumeItem{Teaching students on software engineering practices and the fundamentals of \textbf{Python} and \textbf{OO} design.}
    \resumeItemListEnd
    \resumeSubHeadingListEnd

\resumeSubHeadingListStart
  \resumeSubheading
    {Data Analyst Manager}{Nov 2018 -- Nov 2019}
    {United Parcel Service (UPS)}{Sandy Springs, GA}
    \resumeItemListStart
      \resumeItem{Migrated my team's reporting stack to an automated workflow.}
      \resumeItem{Created dashboards to analyze and aggregate metrics in my division; led to the automation of reports.}
      \resumeItem{Implemented automated reporting applications by designing and rewriting SQL database tables to feed more than 500,000 data entries into VBA and Python applications.}
      \resumeItem{Led and coordinated inter-team reporting, streamlining other teams' workflows; Reduced average reporting workload from 3 hours to 3 minutes using applications I wrote in Python (Pandas, SQLite).}
      \resumeItem{Developed software and mentored peers on three teams using Python, MSQL, and VBA.}
      \resumeItem{Finalists in the UPS upstart startup program; coordinated with a multi-national team to present how encryption technology could impact UPS. Member of the UPS mentorship program.}
    \resumeItemListEnd
    \resumeSubHeadingListEnd

%adding some space%
\vspace{-10pt}


% -----------Multiple Positions Heading-----------
%    \resumeSubSubheading
%     {Software Engineer I}{Oct 2014 - Sep 2016}
%     \resumeItemListStart
%        \resumeItem{Apache Beam}
%          {Apache Beam is a unified model for defining both batch and streaming data-parallel processing pipelines}
%     \resumeItemListEnd
%    \resumeSubHeadingListEnd
%-------------------------------------------


\section{Technical Skills}
{\small
\textbf{Languages:} Java, Kotlin, Python, C, C++, Go, SQL, VBA, HTML/CSS, Bash \\
\textbf{Frameworks} \& \textbf{Libraries:} Spring, JUnit, Gradle, Flask, Android, NumPy, Pandas, OpenCV, Swing, Tkinter \\
\textbf{Dev Tools:} Git, Agile, Google Cloud, AWS, VIM, Docker, Make, Linux, Unit Testing, LaTex \\
\textbf{Misc:} CFA LV 1, NLP, AI, API design, systems design, software architecture, equities, Math club, Seinfeld
}

\end{document}
